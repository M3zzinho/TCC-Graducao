\documentclass[11pt,a4paper]{amsart}

%%% Não mexa aqui

\usepackage{amsmath}
\usepackage{lmodern}
\usepackage[brazil]{babel}
\usepackage{enumerate}
\usepackage[cm]{fullpage}
\usepackage[T1]{fontenc}
\usepackage{natbib}


\usepackage{hyphenat,hyperref}
%\hyphenation{mate-mática recu-perar}

%\usepackage[text={}]{draftwatermark}

\newsavebox{\projetobox}
\newcommand{\projeto}[1]{\savebox{\projetobox}{#1}}

\newsavebox{\alunobox}
\newcommand{\aluno}[1]{\savebox{\alunobox}{#1}}

\newsavebox{\orientbox}
\newcommand{\orientador}[1]{\savebox{\orientbox}{#1}}

\newsavebox{\bancapribox}
\newcommand{\bancapri}[1]{\savebox{\bancapribox}{#1}}

\newsavebox{\bancasegbox}
\newcommand{\bancaseg}[1]{\savebox{\bancasegbox}{#1}}

\newsavebox{\semanobox}
\newcommand{\semano}[2]{\savebox{\semanobox}{#1$^{\underline{\mathrm o}}$
    Semestre de #2}}

\newcommand{\titulo}{
  \noindent
  \centerline{
    \bfseries \Large MAT--0148 --- Introdução ao Trabalho Científico
  }
  \vspace*{1em}
  \noindent
  \centerline{
    \bfseries \large Relatório Final
  }
  \vspace*{1.5em}
}

\newcommand{\cabecalho}{
  \hrule
  \medskip
  \begin{tabular}[h]{ll}
    {\bfseries Título}: &\usebox{\projetobox}\\
    {\bfseries Aluno}: &\usebox{\alunobox}\\
    {\bfseries Orientador}: &\usebox{\orientbox}\\
    {\bfseries Início}: &\usebox{\semanobox}
   \end{tabular}
  \medskip
  \hrule
}

\newcommand{\assinaturas}{
  \setlength{\tabcolsep}{2em}
  \vspace*{.05\textheight}
  \begin{center}
    \begin{tabular}{cc}
      \underline{\hspace*{7cm}}&\underline{\hspace*{7cm}}\\[.5em]
      \usebox{\alunobox}&\usebox{\orientbox}\\
      Aluno&Orientador                        
    \end{tabular}
  \end{center}
}

%%% Fim do não mexa

%%% Pacotes adicionais

% \usepackage[opções]{nome do pacote}

%%% Fim dos pacotes adicionais

%%% Suas marcros aqui

% \newcommand{\meu_comando}{o que faz}
% \renewcommand{\sobrescrever_um_ja_existente}{o que faz}

%%% Fim das suas macros


%%% Dados particulares do seu projeto

\projeto{Fredholm Operators between Hilbert \ensuremath{C^*}-Modules}
\aluno{Gustavo Pauzner Mezzovilla Gonçalves}
\orientador{Severino Toscano do Rego Melo}
\semano{2}{2022}

\orientador{Severino Toscano do Rego Melo -- IME-USP}
\bancapri{Ricardo Bianconni-- IME-USP}
\bancaseg{Gilles Gonçalves de Castro -- UFSC}

%%% Fim dos dados particulares
\usepackage{enumerate}
\usepackage[shortlabels]{enumitem}

\newlist{alter}{enumerate}{1}
\SetLabelAlign{Center}{\hfil#1\hfil}
\setlist[alter,1]{label=(\textit{\alph*}), align=Center}

\newlist{itroman}{enumerate}{1}
\setlist[itroman,1]{label=\ensuremath{(\roman*)}}

\begin{document}

\titulo{}
\cabecalho{}

\section{Resumo }

Durante o ano letivo de 2022, o aluno realizou um estudo detalhado de todas as componentes principais do artigo ``\textit{A Fredholm Operator Approach to Morita equivalence}'', de Ruy Exel \cite{exel7fredholm} contemplando tópicos de $K$-teoria de Álgebras de Banach, \ensuremath{C^*}-Módulos de Hilbert, Operadores de Fredholm generalizados e suas aplicações. 

\section{Execução, Andamento e Objetivos Atingidos}

A metodologia utilizada foi a usual na pesquisa em Matemática, com exposição dos resultados obtidos pelo aluno e discussão dos mesmos. Os temas abordados no plano inicial estão apresentados a seguir, com o respectivo parecer.
\begin{itroman}
	\item \textit{Estudar módulos de Hilbert sobre $C^*$-algebras.} 
	\item[] $\triangleright$ Realizado.
	\item \textit{Estudo de operadores entre Módulos de Hilbert; Generalização dos conceito de posto e compacidade desses operadores; Estudar módulos quase isomorficamente estáveis e classificação dos mesmos.}
	\item[]$\triangleright$ Realizado. O aluno pode reunir diversos exemplos e apresentar de uma forma detalhada a demonstração do Teorema 2.7 do artigo de referência \cite{exel7fredholm}, além de simplificar um argumento. Os conteúdos necessários nessa fase se acumularam de modo que o aluno necessitou estudar o Teorema de Estabilização de Kasparov \cite{kasparov1980stinespring} e mais propriedades do mapa do índice $\partial : K_1(A/I)\longrightarrow K_0(I)$. Além disso, algumas interações com o autor foram necessárias para uma exposição mais detalhada.

  \item \textit{Estudar operadores de Fredholm entre $C^*$-módulos de Hilbert sobre $A$; Definição do índice de Fredholm ind : $\mathcal{L}_A(E, F) \longrightarrow K_0(A)$; Interpretação e comparação dos resultados clássicos nessa abordagem.}
	\item[] $\triangleright$ Realizado. Muitos resultados técnicos necessitaram um estudo mais detalhado, de modo que fontes auxiliares foram necessárias, além de interações com o autor. Vale mencionar que durante a realização do projeto, o objetivo final sofreu alteração, passando a ser explicitar as relações dos operadores de Fredholm.

  \item \textit{Estudo de representações de bimódulos de Hilbert e suas relações com $K$-teoria; Compreender o resultado final de \cite{exel7fredholm} a respeito do isomorfismo entre os $K$-grupos de $C^*$-álgebras Morita-Rieffel equivalentes; Observar as diferenças teóricas entre este e o resultado obtido por Brown, Green e Rieffel por meio da abordagem dos operadores de Fredholm.}
	\item[] $\triangleright$ Frente a mudança de objetivo, o teorema de Brown, Green e Rieffel que era o objetivo inicial não foi estudado em detalhes da mesma forma que os tópicos anteriores. Estes representaram parte substancial do desenvolvimento do projeto, e, não necessitou-se compreender afinco o estudo de representações devido. Apesar da comparação entre as técnicas não ser realizada, visto que, o estudo da demonstração original requisitaria outros campos de estudo, a vantagem do método fora ressaltada quando explicitada a ausência da hipótese de $C^*$-álgebra $\sigma$-unitais durante os temas anteriores.

  \item \textit{Caso seja possível, o aluno irá investigar as relações entre o índice de Fredholm e o índice Atiyah para famílias contínuas de operadores de Fredholm.}
  \item[] $\triangleright$ Tópico extra não realizado.
\end{itroman}

%\section{Objetivos Antingidos}

%Uns poucos parágrafos relatando de maneira resumida os principais resultados estudados, comparando isso com o cronograma previsto no Projeto Inicial, mencionando alterações de planos e justificativas para essas alterações etc.

\section{Composição da Banca Examinadora}

A banca examinadora da monografia é comporta por:

\bigskip
\begin{enumerate}[1.]
  \item \usebox{\orientbox}
  \item \usebox{\bancapribox}
  \item \usebox{\bancasegbox}
\end{enumerate}
\nocite{brown1977morita,exel7fredholm,kasparov1980stinespring,raeburn1998morita,rieffel1981c,RIEFFEL1974176}

%\assinaturas

\bibliographystyle{plain}
\bibliography{ref}

\end{document}

%%% Local Variables: 
%%% mode: latex
%%% TeX-master: t
%%% End: 
