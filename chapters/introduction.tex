\chapter*{Introduction}

\section*{Contextualization}

The present document represents a detailed study of the author in studying the article ``A Fredholm Operator approach to Morita Equivalence'' by Ruy Exel. In it, the theory of generalized Fredholm operators between Hilbert $C^*$-modules is developed, and a version of the Brown, Green and Rieffel theorem concerning induced K-groups on equivalent Morita-Rieffel $C^*$-algebras is demonstrated. This includes all the definitions and theorems necessary to understand the machinery developed by Exel in that article. In addition, it includes all the comments that I deemed relevant during my understanding of these topics.

The original theorem (henceforth, BGR) does not concern $K$-theory, but the connection between Morita-Rieffel equivalence and stably compact isomorphic relation between $A$ and $B$, i.e., if $\mathscr K$ is the algebra of compact operators on a separable Hilbert space, then $A\otimes \mathscr K \simeq B\otimes \mathscr K$. The theorem quoted above:
\begin{quote}
\textbf{Theorem} (BGR - \cite[Theorem 1.2]{brown1977morita}) 
Let $A$ and $B$ be $C^*$-algebras. If $A$ and $B$ are stably compact isomorphic, then they are strongly Morita equivalent. Conversely, if $A$ and $B$ are Morita-Rieffel equivalent and if they both possess strictly positive elements (i.e., they are $\sigma$-unital), then they are stably compact isomorphic.
\end{quote}
A corollary of BGR is that $\sigma$-unital Morita-Rieffel algebras induce the same $K$-groups. In the original concept, the separability of the algebras is crucial for this result, given that the authors themselves found cases in which the theorem doesn't hold. However, the same question about induced $K$-groups was open when one abstains from separability of algebra. Besides that, the original proof guarantees that $A\otimes \mathscr K$ and $B \otimes \mathscr K$ are isomorphic, but without any clue on what this isomorphism should be. Hence, there isn't any explicit expression to the induced $K$-theoretic isomorphism.

In his work, Ruy Exel exhibited explicit isomorphisms $K_0(A)\longto K_0(B)$ and $K_1(A)\longto K_1(B)$ for two $C^*$-algebras $A$ and $B$ Morita-Rieffel equivalents, dribbling the separability hypothesis. His doing occurs treating the mentioned Fredholm operators as Mingo's approach \cite{mingo1987K} and obtaining the abelian group $F(A)$ of all $A$-Fredholm operators up to containing the same index. 

This is a work of a graduation student, who wanted to facilitate further studies in those topics, with a more unified reference containing all the necessary ingredients. I encourage any reader to treat this document as a journey through many faces of mathematics, and, to obviously check the sources of the material. 

\section*{Itinerary}
Here, we focus on detailing the construction of those Fredholm operators and the predecessor steps necessary. Therefore our road map can be described as follows:
\begin{itroman}
	\item \underline{\textit{$K$-theory for Banach Algebras}}: Since the main result is a $K$-theoretic one and, the index of our $A$-Fredholm operators will be elements of $K_0$, Chapter \ref{ch:k theory} is devoted to contain a minimalist toolkit bag, which is the definitions of the abelian groups $K_0(A)$ and $K_1(A)$ for a Banach algebra $A$, and the index map induced in a short exact sequence. Some famous properties about $K$-theory of operator algebra will be mentioned, but only as snacks at boring adult parties.
	\item \underline{\textit{Hilbert and Finite-rank \ensuremath{C^*}-Modules}}: In order to define the so called Fredholm operators between Hilbert $C^*$-modules, naturally, one needs to understand what those modules are, were they live and what do they feed. The initial sections in Chapter \ref{ch:hilbert modules} tackles abstract versions of these questions in a mathematical sense. In particular, one needs to understand a new notion of compactness and the so called \textit{finite-rank} modules described in Section \ref{sec: finite rank operators}. Those will replace our notion of finite-dimensional. One of the greatest technical theorems used in Exel's arguments is the Kasparov Stabilization Theorem \ref{teo: kasparov stabilization} presented in Section \ref{sec: kasparov stabilization}, beeing crucial to discarding the separability assumption in the target theorem and the definition of Fredholm operators, which will require a classification like theorem \ref{teo: M N quasi stably iso ==> rank igual} of the so called Quasi-Stably-Isomorphic Finite-Rank Hilbert $C^*$-modules, to whose the last Sections of Chapter \ref{ch:hilbert modules} are dedicated. The purpose of these Chapter is to be a more detailed Section 2 of Exel's paper \cite{exel7fredholm}.
	
	\item \underline{\textit{Generalised Fredholm operators}}: The big star of the show, as you may anticipate by the title of this work, Chapter \ref{ch:fredholm operators} is the place where they're treated. Since our Fredholm operators don't need to have closed range, typical arguments involving projections can't be used. This will lead us to define in Section \ref{sec: regular fredholm} a ``smaller'' class, called the \textit{regular} operators, whose index can be defined. Later, it'll be showed that every general Fredholm operator can be regularized. Finally, we finish the Chapter constructing a Fredholm portrait of $K_0(A)$, giving birth to an abelian group $F(A)$ and a isomorphism $\ind : F(A) \longto K_0(A)$.
	
	\item \underline{\textit{A sketch of Exel's application to Morita Equivalence}}: Up to this point, there's already too much to digest. Therefore, we discuss briefly what Morita-Rieffel equivalence is, and present only the skeleton of the arguments used by Exel in the extension of BGR, quoting without proving its main results.      
\end{itroman}

All sections are filled with some examples and some historical context which I believe gives useful perspective. 

