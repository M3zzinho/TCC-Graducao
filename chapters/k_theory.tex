\chapter[\texorpdfstring{\ensuremath{K}}{K}-theory of Banach Algebras]{\texorpdfstring{\mathbf{K}}{K}-theory of Banach Algebras}
\label{ch:k theory}

\section{General portrait of homological theories}
%- A homological theory is ...
%- Some flavors of homological theory
%- K-theory is our concern
A homological theory for a category $\mathcal C$ consist in a sequence of covariant functors $H_n : \mathcal C \longto \Ab{\Grp}$ for each $n\in \N$ which satisfies some set of axioms, which depends on what theory one is interested. For example, if $\mathcal C$ contains a nice homotopical concept, it's rather common to ask for homotopical invariance. If exact sequences naturally pops in the domain encoding a lot of information, some other axioms are required to obtain long exact sequences. The usual notation is:
\begin{equation*}
    \describefunctor{H_n}{\mathcal{C},A,\phi,B}{\Ab{\Grp},H_n(A),\phi_n,H_n(B)}
\end{equation*}
We also need a way to translate short exact sequences of the form
$$0 \longto A \longto B \longto C \longto 0$$ 
from the original category to higher counterparts obtained by $H_n$, hence, every homology theory seeks to define a connecting morphism $\delta_n : H_n(C) \longto H_{n+1}(A)$ into a long exact sequence: 
\begin{equation*}
    \begin{tikzcd}
  H_0(A) \rar & H_0(B) \rar 
             \ar[draw=none]{d}[name=X, anchor=center]{}
    & H_0(C) \ar[rounded corners,
            to path={ -- ([xshift=2ex]\tikztostart.east)
                      |- (X.center) \tikztonodes
                      -| ([xshift=-2ex]\tikztotarget.west)
                      -- (\tikztotarget)}]{dll}[at end]{\sub\delta0}  \\
  H_1(A) \rar & H_1(C) \rar 
             \ar[draw=none]{d}[name=Y, anchor=center]{}
    & H_1(B) \ar[rounded corners,
            to path={ -- ([xshift=2ex]\tikztostart.east)
                      |- (Y.center) \tikztonodes
                      -| ([xshift=-2ex]\tikztotarget.west)
                      -- (\tikztotarget)}]{dll}[at end]{\sub\delta1} \\ 
    H_2(A) \rar & H_2(B)\rar  & \cdots  \\
\end{tikzcd}
\end{equation*}

In the other hand, as everything containing the prefix ``co'', cohomology theories are consisted of contra-variant functors $(H^n)_{n}$ with the same pay-off. The position of the index on the notation usually indicates what sort of theory one is dealing with. 

Here we are concerned with a homology theory for complex Banach Algebras $\BAlg$ or, more popularly, for $C^*$-algebras $\Cstar$, a.k.a., $K$-theory for Operator Algebras. It is the mirror image of Topological $K$-theory, in light of \textit{Gelfand Duality} connecting the category of Locally Compact Hausdorff spaces and complex abelian $C^*$-algebras, but not restricted to commutative spaces, 
which is often referred to as the "Non-Commutative Topology". 

In his work to reformulate Riemman-Roch theorem \cite{borel1958theoreme}, \textit{A. Grothendieck} introduces the group $K(A)$ associated to a subcategory of an abelian category, which nowadays, it is the so called ``Grothendieck's group''. That's where is from the letter $K$, which he had chosen for ``Klassen''. His reformulation famously contains his legendary drawing:
\begin{figure}[H]
	\centering
	\input{figures/grothendieck-riemann-roch.tex}
\end{figure}
\begin{quote}
	\textsl{Riemann-Roch Theorem: the new black: The diagram}
	$$[\ldots] \textsl{ is commutative!}$$
	\textsl{I would need to misuse about $2 h$ of my listener's time in order to impart only an approximal understanding of this statement for }$f:X \longrightarrow Y$.
	
	\textsl{In cold print (as in Springer's lecture notes) this would take around $400$-$500$ pages.}
	
	\textsl{A thrilling example of how our urge for knowledge and discoveries decays into a lifeless and ideological delirium while life itself goes thousandfold to the devil - and is threatened with final destruction.}
	
	\textsl{It's high time to change our course!}

	$(16.12 .1971)$ \hfill Alexander Grothendieck
\end{quote}
With his work settled in what is about to be algebraic $K$-theory, topological $K$-theory would be a product of \textit{M. Atiyah} and \textit{F. Hirzebruch} replicating Grothendieck's construction for topological vector bundles over compact Hausdorff spaces. 

We'll construct functors $K_n:\BAlg \longto \Ab{\Grp}$ and the connecting maps will be call \textit{index map}, denoted by $\partial$. A remarkable aspect of operator $K$-theory is the \textit{Bott periodicity}: $K_n \simeq K_{n+2}$, which then describes for any short exact sequence $0 \longto I \longto A \longto A/I \longto 0$, where $I\triangleleft A$, a six-term exact sequence:
\begin{equation*}
    % https://tikzcd.yichuanshen.de/#N4Igdg9gJgpgziAXAbVABwnAlgFyxMJZABgBpiBdUkANwEMAbAVxiRAGkB9YgCgEkAlCAC+pdJlz5CKAIzkqtRizZdeAQSGjx2PASIAmedXrNWiDtwA6lhjABmOHtYC2dgE50AxgAI1fa25YAOYAFjiaYiAYOlJEZDIKJsrmXDLWtg5Olq4ePn4BwWER2pJ6sqQJxkpmFjI8GiKR0aXSyIaViqYqnHWCIgowUEHwRKDuEM5IZCA4EEgyWiDjk4hyM3OI+ovLSIbrSACsVV3m1mh0bniMjWNuE4fUs0gALNt3K8+PGwDMb-eI3y+U2OyRAZwuVwY-WEQA
\begin{tikzcd}
K_0(I) \arrow[r]                                & K_0(A) \arrow[r] & K_0\left(A/I\right) \arrow[d, "\partial"] \\
K_1\left(A/I\right) \arrow[u, "\partial"] & K_1(A) \arrow[l] & K_1(I) \arrow[l]                               
\end{tikzcd}
\end{equation*}

The details will be spared in what is outside of our scope, which will include the definition of the groups $K_0$ and $K_1$ for complex Banach algebras, and the index map mentioned. The connecting map will be used in the classification of finite rank modules, and later on, the index of our Fredholm operators. Hence it is important to define it in a helpful way. 


\section[The \texorpdfstring{\ensuremath{K_0}}{K0} group]{The \texorpdfstring{\mathbf{K_0}}{K0}-group}
%- Quote topological counterparts
%- Define $K_0$
%- Summon properties
%- Some examples 
Our object is to deal with Hilbert $C^*$-modules, witch are right $A$-modules with a generalized $A$-valued inner product for a given $C^*$-algebra $A$ (plus some other details), generalizing the concept of Hilbert space. Therefore, it is reasonable to understand some $K$-theory for $C^*$-algebras, as they are our underlying space. Unfortunately, as we'll see later on, there is no \textit{Riesz representation lemma} and, there exists bounded linear operators that aren't adjointable between Hilbert modules. Hence, dealing with self-adjoint operators is a restriction for sure. Thankfully, the $K$-theory for Banach algebras is good enough in order to fill our needs.

In topological $K$-theory, in order to define the $0^{\underline{\text{th}}}$ $K$ group, one would consider a complex vector bundle $E$ over $X \in \CHaus$ and take the right $C(X)$-module $\Gamma(X,E)$ of continuous sections $s:X\longto E$ with pointwise scalar multiplication\footnote{That is to say, for $f\in C(X)$ and $s\in \Gamma(X,E)$, let $x\longmapsto s(x)f(x)$.}. Compactness of $X$ implies that $\Gamma(X,E)$ is a projective $C(X)$-module, and \textit{Serre-Swan} theorem  \cite[Theorem 6.18.]{karoubi2008k} states that $E\longmapsto \Gamma(X,E)$ induces an equivalence between the category of complex vector bundles and finitely generated projective $C(X)$-modules. Hence, $K^0(X)$ is the \textit{Grothendieck group} of the set of equivalence classes of all isomorphisms between vector bundles over $X$.

For a given Banach Algebra $A$, the following definitions and constructions mimics the above paragraph, by replacing vector bundles by finitely generated projective $A$-modules. 

\begin{definicao}
\label{def: idempotentes equivalentes}
In any given Banach algebra $A$, for two idempotent elements $x$ and $y$, define the following notions of equivalence: 
\begin{itroman}
\item \textbf{Murray-von-Neuman equivalent}: There are elements $p,q\in A$ such that $x=pq$ and $y=qp$.
\item \textbf{Similarly equivalent}: There exists an invertible\footnote{Assuming that $A$ is unital.} element $u\in \GL(A)$ such that $x = \inv u y u$.
\item \textbf{Homotopic}: There are a continuous path  $\gamma \in C([0,1], A)$ of idempotents between $x$ and $y$, i.e.,
\begin{equation*}
    \gamma(0) = x, \gamma(1)=y \e \forall\,t\in [0,1], \gamma(t)^2 = \gamma(t).
\end{equation*}
\end{itroman}
If $A$ is assured to be a $C^*$-algebra, those definitions are concerned with self-adjoint idempotent elements, a.k.a., projections. Two projections $x, y$ are equivalent if there exists a \textit{partial isometry} $u$ such that $x= u^*u$ and $y=uu^*$.
\end{definicao}



For the canonical embedding $x \longmapsto \diag(x,0)$ over matrices, consider the inductive limit $\mathbb M_\infty(A) \coloneqq \varinjlim_{n\in \N} \mathbb M_n(A)$, which can be seen as the set of infinite matrices over $A$ but only finitely many of the entries are non-zero. 

\begin{observacao}
\label{obs: similar em M_inf}
Note that $\mathbb M_\infty(A)$ contains no unity, but that doesn't stop us to declaring two elements $x,y$ to be similar when they are similar in some square matrix space $\mathbb M_n(A)$. Therefore, all equivalence relations listed in the definition \ref{def: idempotentes equivalentes} coincide in $\mathbb M_\infty(A)$.
\end{observacao}

Simply shouting ``Let $A$ be an $C^*$-algebra'' in the crowd is a powerfull classification tool, whenever is a mathematicians crowd\footnote{Otherwise, you are just playing creepy at dinner table again.}. 
\begin{itroman}
    \item If you hear in response ``unital or not?'', you know that there is some $C^*$-algebraic fellow around you.
    \item If the crowd contains mathematicians and no-one asks if $A$ contains a unity or not, no $C^*$-algebraist is contained in the crowd. They are instantly assuming the unity is there. 
\end{itroman}

This is because dealing without unital rings outside $C^*$-theories are usually simple. Just unitize and go on. However, the presence of unity in $C^*$-algebras is crucial to determine their underlying hidden topology, as explicitly is made in \textit{Gelfand's duality} theorem.

The next definition is in charge to define the functor $K_0$ for both cases, but some intermediate steps are required from one to another.

\begin{definicao}
Let $A$ be a Banach algebra. The set of equivalence classes over $\mathbb M_\infty(A)$ considering any relation $\sim$ contained in \ref{def: idempotentes equivalentes} is an abelian semi-group with $[x]+[y] \coloneqq [\diag(x,y)]$. Before defining $K_0$, in order to include the non necessarily unital algebras, it is needed to be considered an auxiliar functor $K_{00}$ much closer to the topological counterpart $K^0$. This is necessary in order to obtain the Bott periodicity result for Banach algebras, and other good functorial properties.

\begin{itroman}
\item \textbf{\mathbf{K_{00}}}: It is the Grothendieck group construction associated with the semi-group $V({A}) \coloneqq \mathbb M_\infty\left({A}\right)/{\sim}$ where addition is given by $[x]+ [y] \coloneqq [\diag(x,y)]$, generalising the construction of $\Z$ from $\N$ considering formal differences. In lighter sheets, for pairs $(a,b)$ and $(c,d)$ of elements in $V(A)$, let $(a,b) \equiv (c,d)$ whenever there exists\footnote{Since it is only a semi-group, the cancelation property do not hold necessarily over $V(A)$. One might check that this is the case if, and only if, the inclusion of $V(A)$ at the Grothendieck's associated group is injective.} $z \in V(A)$ such that $a+d+z = c+b+z$. This is an equivalence relation over the pairs, and $[\come]_{00}$ will denote the related equivalence class. 

We are mimicking the formal differences construction, so it's natural to define the addition operation coordinate-wise and let $x-y \coloneqq [(x,y)]_{00}$. Therefore, it is well defined the following covariant functor:
\begin{equation*}
    \covfunctor{{K_{00}}{\BAlg}{\Ab{\Grp}}{A}{\mfrac{V(A)\times V(A)}{\equiv}}{B}{\mfrac{V(B)\times V(B)}{\equiv}}{\phi}{\sub\phi{00}}{x-y}{\phi(x)-\phi(y)}}
\end{equation*}

Since every element in $V(A)$ is the class of some idempotent matrix $p$, we can state that every element in $K_{00}(A)$ is on the form $[p]_{00} - [q]_{00}$. Two formal differences $[p]_{00}-[q]_{00}$ and $[x]_{00}-[y]_{00}$ coincide in $K_{00}(A)$ precisely when the operators $\diag(p,y)$ and $\diag(x,q)$ are \textit{stably} homotopic. 
\item \textbf{\mathbf{K_0}}: In our next step, it's crucial to know exactly who $K_{00}(\C)$ is. Hence, remember that two idempotents in $\mathbb M_n(\C)$ are similar if, and only if, their images has the same dimension. Therefore $V(\C) \simeq \N$, and by historical nightmares with Analysis I exercise constructing the integer numbers, it is easy to infer that $K_{00}(\C) = \Z$. 

For non necessarily unital $A$, consider $\widetilde{A} \coloneqq A \oplus \C$ the \textit{unifization} of $A$ and the complex projection $\varepsilon: \widetilde{A} \longtwoheadrightarrow \C$, which induces the short exact sequence:
\begin{equation*}
    \label{eq:short exact sequence unital}
\begin{tikzcd}
    0 \arrow[r] & A \arrow[r, hook] & \widetilde A \arrow[r, "\varepsilon", two heads] & \mathbb C \arrow[r] & 0
    \end{tikzcd}
\end{equation*}
The urge to obtain Bott periodicity theorem for Banach algebras, which is a relation between $K_0$ and $K_1$ in the presence of short exact sequences, will obligate the exactness of the following:
\begin{equation*}
    \label{eq:K0(short exact sequence unital)}
\begin{tikzcd}
    0 \arrow[r] & K_0(A) \arrow[r, hook] & K_0(\widetilde A) \arrow[r, "\sub\varepsilon0", two heads] & K_0(\mathbb C) \arrow[r] & 0
    \end{tikzcd}
\end{equation*}
Since it is a morphism between unital Banach algebras, the induced map $\sub{\ep}{00} : K_{00}(\widetilde{A}) \longto \Z$ is a well defined morphism, hence, it is possible to define the following:
\begin{equation*}
    \covfunctor{{K_0}{\BAlg}{\Ab{\Grp}}{A}{\ker(K_{00}(\widetilde{A}) \to \Z)}{B}{\ker(K_{00}(\widetilde{B}) \to \Z)}{\phi}{\sub\phi0}{a+ z}{\phi(a)+z}}
\end{equation*}
Notice that $K_0(A)$ is precisely the set of elements $[p]_0-[q]_0 \in K_{00}(\widetilde{A})$ such that $\ep(p) \sim \ep(q)$. If $A$ is already unital, it is possible to show that $K_0(A) \simeq K_{00}(A)$. 
\end{itroman}
\end{definicao}

\begin{observacao}
    The argument to show that $V(\C) \simeq \N$  is equivalent for compact operators in a infinite-dimensional Hilbert space $H$, i.e. $V(\mathscr K(H)) \simeq \N$, hence $K_0\mathscr K(H) = \Z$. On the other hand, any two infinite rank projections in $\mathscr B(H)$ are equivalent, hence $V\mathscr B(H) \simeq \N \cup \{\infty\}$, which is a semi-group without the cancellation property. Since everyone is equivalent to $\infty$, it is obtained that $K_{00}\mathscr B(H) \simeq 0$. The semi-group $V(A)$ posses the cancellation property if, and only if, the inclusion $V(A)\longhookrightarrow K_{00}(A)$ is injective. 
\end{observacao}

\begin{proposicao}[Standard portrait of \ensuremath{K_0}]
Every element of $K_0(A)$ can be written as $[x+p_n]_0-[p_n]_0$.

\begin{proof}
Let $p, q \in \mathbb M_{\infty}(\widetilde A)$ be some idempotent square matrices with order no longer than $n$, such that $\sub\ep0\left([p]_0 - [q]_0\right) = 0$, i.e. $[p]_0 - [q]_0 \in K_0(A)$. Matrices $p \in \mathbb M_n(\widetilde{A})$ can be written as $(\sub pA, \sub p{\C}) \in \mathbb M_n(A)\oplus \mathbb M_n(\C)$, i.e. an algebraic part $\sub pA$ and a scalar part $\sub p{\C}$. Stating that  $\sub\ep0\left([p]_0 - [q]_0\right) = 0$ means that the scalar parts of $p$ and $q$ coincide.

The identity $\Id_n \in \mathbb M_\infty(\widetilde{A})$ can be seen as the projection operator of the first $n$-th coordinates, by filling it with 0's, but to avoid confusions, let it be $p_n$. With $y\leqslant x$ be given by $xy = yx = y$, one may see that $p < p_n$ and $y < p_n$. Notice that
$\diag(0,p) \in \mathbb M_{2n}(\widetilde{A})$ is similar to $p$ and orthogonal to $\Id_n$, i.e. 
\[
\begin{pmatrix}
    0 & 0 \\ 0 & p
\end{pmatrix}
\begin{pmatrix}
    \Id_n & 0 \\ 0 & 0 
\end{pmatrix} = 
\begin{pmatrix}
    \Id_n & 0 \\ 0 & 0 
\end{pmatrix} \begin{pmatrix}
    0 & 0 \\ 0 & p
\end{pmatrix} = 0.
    \] 
Hence, $x \coloneqq \diag(-q,p)$ is such that $x+p_n$ is an idempotent operator and:
\begin{equation*}
    \begin{array}{rcl}
        [x+p_n]_0 - [p_n]_0 &=& \left[\diag(0,p)\right]_0 + \left[p_n - q\right]_0 - \left(\left[p_n-q\right]_0 + \left[q\right]_0\right) \\
        &=& \left[p\right]_0 - \left[q\right]_0. 
    \end{array}\hfill \qedhere
\end{equation*}
\end{proof}
\end{proposicao}

\section[The \texorpdfstring{\ensuremath{K_1}}{K1} group]{The \texorpdfstring{\mathbf{K_1}}{K1}-group}
While $K_0$ is build upon equivalence classes of idempotent, $K_1$ uses invertible elements, but simpler. Therefore, let $\GL_\infty(A) \coloneqq \varinjlim_{n\in\N} \GL_n(A)$ considering the embedding $x \longmapsto \diag(x, 1)$. Calculus is back, and we shall consider exponentials inside a unital algebra $A$:
\begin{eqspaced*}{(a\in A)}
    \exp(a) \coloneqq \sum_{n=0}^\infty \dfrac{a^n}{n!} \e \log(1+a) \coloneqq \sum_{n=1}^\infty -\dfrac{{a}^n}{n} 
\end{eqspaced*}
\hspace{-0.15cm}where the log is defined whenever $\|a\| < 1$. This is the case since elements of the form $z-a$ for complex $z$ are invertible if $\|a\| \leqslant |z|$. If $a$ and $b$ doesn't commute, $\exp(a)\exp(b)\neq \exp(a+b)$, which means that the set of exponentials aren't closed by multiplications.

\begin{lema}
    \label{lema:exp(A)=GL(A)_0}
For a unital Banach algebra $A$, the component of the unity is the group generated by $\{\exp(a) \mid a \in A\} \subset \GL(A)$, denoted by $\exp(A)$.

\begin{proof}
    Let $\GL^{(0)}(A)$ be the refered set of connected components of $1$. Notice that $t \longmapsto \exp(tb)$ for $t\in [0,1]$ is a continuous path of invertible elements between $1$ and $\exp(b)$ for any $b$, hence $\exp(A) \subset \GL^{(0)}(A)$. It remais only to show the converse inclusion.

    For some $a$ with $\|1-a\| < 1$, let $b\coloneqq \log(1 + (a-1)) = \log(a)$, i.e. $a = \exp(b)$. Therefore, if $u\in \GL(A)$ and $\|v-u\| < \inv{\|\inv u\|}$, this means that $v= \exp(b)u$ for some $b$. From this treatment, if follows that $\exp(A)$ is a open and closed topological subspace of $\GL^{(0)}(A)$ which contains the unity, i.e. $\GL^{(0)}(A)$ coincides with $\exp(A)$. 
\end{proof}
\end{lema}

\begin{observacao}
    Let $M\in \GL_n(\C)$. Since $0$ cannot be an eigenvalue of $M$ (which is a finite set), it's possible to find $\alpha \neq 0$ such that $[0, \infty)\cdot \alpha$ doesn't contains any of the eigenvalues of $M$ or $1$. Therefore, $1-\alpha t \neq 0$ for all $t\geq 0$ and $M_t \coloneqq \inv{(1-\alpha t)}(M-t\alpha \Id_n)$ is a continuous path from $M$ to the identity, i.e. $\GL_n(\C)$ is connected. 
\end{observacao}


In a not so long future, the following result will be important in the presence of an ideal $I\triangleleft A$, the considering of the projection $A \longtwoheadrightarrow A/I$.
\begin{corolario}
    \label{corol:lift by surjection}
    Any continuous surjection $A \longto B$ induces a lift from every element in $\GL_n^{(0)}(B)$ to one in $\GL_n^{(0)}(A)$.
    \begin{proof}
        Using \ref*{lema:exp(A)=GL(A)_0}, write $\prod_i \exp(b_i) \in \GL_n(B)^{(0)}$ for any desired element. Since there is a surjection, there exists lifts to each $b_i$, i.e. $a_i \in \GL_n(A)$ such that $\prod_i \exp(a_i) \in \GL_n(A)_{0}$.
    \end{proof}
\end{corolario}

Considering the homotopy equivalence relation, two elements in $\GL_\infty(A)$ are homotopical whenever they are in the same connected component in some $\GL_n(A)$. Denote the equivalence class by $[\come]_1$. Whence, the quotient $\GL_\infty(A)/\GL_\infty^{(0)}(A)$ is an abelian group with the multiplication $[x]_1[y]_1 = [xy]_1$, which is commutative once you note it is possible to find a connected path\footnote{
        Let $z(\theta) = \left(\begin{smallmatrix}
            \cos \theta & -\sin \theta \\ \sin \theta & \cos \theta
        \end{smallmatrix}\right)$ be the rotation matrix by some angle $\theta$. Therefore, the continuous map $[0,\pi/2] \ni \theta \longto z(\theta) \diag(y,1) \inv{z(\theta)}$ is the desired path.
    } between $\diag(y,1)$ and $\diag(1,y)$, hence 
    \begin{equation*}
        [x]_1[y]_1 = [xy]_1 = \left[\begin{pmatrix}
            xy & 0 \\ 0 & 1
        \end{pmatrix}\right]_1= \left[\begin{pmatrix}
            x & 0 \\ 0 & 1
        \end{pmatrix}\begin{pmatrix}
            y & 0 \\ 0 & 1
        \end{pmatrix}\right]_1 = \left[\begin{pmatrix}
            x & 0 \\ 0 & y
        \end{pmatrix}\right]_1 
    \end{equation*}    
    and similarly, one shows that $[xy]_1 = [\diag(y,x)]_1 = [y]_1[x]_1$. There we have, our $K_1(A)$ group. Since $\GL_n(\C)$ is connected, it follows immediately that $K_1(\C) = 0$ and, therefore, we can deal with units the way it is intended: for non necessarily unital algebras $A$, let $K_1(A) \coloneqq K_1(\widetilde{A})$.

\begin{definicao}
    The functor $K_1$ can be seen as the following:
    \begin{equation*}
        \covfunctor{{K_1}{\BAlg}{\Ab{\Grp}}{A}{\mfrac{\GL_\infty(\widetilde A)}{\GL_\infty^{(0)}(\widetilde A)}}{B}{\mfrac{\GL_\infty(\widetilde B)}{\GL_\infty^{(0)}(\widetilde B)}}{\phi}{\sub\phi1}{[x]_1}{[\phi(x)]_1}}
    \end{equation*}
\end{definicao}

\begin{observacao}
    It should be stated that if one is dealing with a $C^*$-algebra, then $K_1$ can be obtained by the set of unitary matrices $U_n(A)$, i.e. $u^* = \inv u$; Since $U_n(A)/U_n^{(0)}(A) \cong \GL_n(A)/\GL_n^{(0)}(A)$, one can obtain a deformation retraction from $\GL_n(A)$ to $U_n(A)$ by the polar decomposition, hence, $K_1(A)$ is isomorphic to $U_\infty(A)/U_\infty^{(0)}(A)$.  
\end{observacao}

%- Quote topological counterparts
%- Define $K_1$
%- Summon properties
%- Some examples

\section{The index map}

Many functorial properties like homotopy invariance, stability and half exactness are shared by $K_0$ and $K_1$, but in our needs, they wont be necessary. The reader may recall to \cite{blackadar1998k} or \cite{wegge1993k} to a proper course.

We are now ready to define the so called index map. This name comes from the Fredholm operator theory since what we are about to construct is a generalization of the index of those operators. Consider $\mathscr B(H)$ the $C^*$-algebra of bounded operators between a Hilbert space $H$, and $\mathscr K(H)$ the ideal of compact operators. The \textit{Atikinson} theorem states precisely that the \textit{Calkin} algebra $\mathscr Q(H) \coloneqq \mathscr B(H)/\mathscr K(H)$ is a classifying one: $T$ is a Fredholm operator if, and only if, $(T \bmod \mathscr K(H)) \in \GL \mathscr Q(H)$.

Since $K_0\mathscr K(H) = \Z$ and $K_1\mathscr Q(H)$ can be seen the set of Fredholm operators up to homotopy\footnote{Remind that two Fredholm operators in the same realm have the same index if, and only if they are homotopic.}, the index map $\ind : K_1\mathscr Q(H) \longto K_0\mathscr K(H)$ is well defined. Our index map $\partial$ will generalize this application. 

\begin{construcao}
\label{contru: index map}
    Let $I\triangleleft A$ and consider the following short exact sequence:
    \begin{equation*}
        \begin{tikzcd}
            0 \ar[r] & I \arrow[r, hook] & A \arrow[r, two heads] & \mfrac AI \ar[r] & 0
        \end{tikzcd}
    \end{equation*}
    We are in position to construct $\partial : K_1(A/I) \longto K_0(I)$. For $[x]_1 \in K_1(A/I)$, let $n$ be such that $x \in \GL_n(\widetilde{A/I})$. It's  about time to the corollary \ref{corol:lift by surjection} to shine: Since the projection $A \longtwoheadrightarrow A/I$ is a continuous surjection, so it is the unifisation induced morphism between the algebras, hence, one can lift the element $\diag(x, \inv x) \in \GL_{2n}^{(0)}(\widetilde{A/I})$ to some $w \in \GL_{2n}^{(0)}(\widetilde A)$.
    
    If $\pi : \GL_\infty(\widetilde{A}) \longtwoheadrightarrow \GL_\infty(\widetilde{A/I})$ is the quotient projection, notice that $\pi(wp_n\inv w) = p_n$, so that $wp_n\inv w \in \widetilde{I}$. Since $wp_n \inv w$ is also an idempotent, notice that $[wp_n \inv w]_0 - [p_n]_0 \in K_0(I)$. And this is the image of the index map $\partial$ of some element $[x]_1$.
\end{construcao}

An anxious mind would immediately panic. We have a \textsc{to-do} list before calling the day:
\begin{itroman}
    \item\label{todo:(i)} Check that $[wp_n \inv w]_0 - [p_n]_0$ doesn't depend on the lift $w$ chosen;
    \item\label{todo:(ii)} Check that $\partial([x]_1) = \partial([y]_1)$ for $x \equiv y$.
    \item\label{todo:(iii)} Check that $\partial$ is a group morphism.   
\end{itroman}
\begin{proof}[Proof of \textsc{to-do} list items]
    If $v$ is another lift of $\diag(x, \inv x)$, notice that 
    $$
    vp_n \inv v = (v\inv w)wp_n\inv w\inv{(v\inv w)},
    $$
    i.e. $vp_n \inv v$ is similar to $wp_n \inv w$. This is enough to take care of \ref{todo:(i)}.

    In order to show that the index is well defined, suppose that $y \in \GL_n(\widetilde{A/I})$ is another representant of class $[x]_1$. Notice that 
    $$
    \inv xy \in \GL_n^{(0)}(\widetilde{A/I}) \e \begin{pmatrix} x & 0 \\ 0 & \Id_n\end{pmatrix}\begin{pmatrix} \Id_n & 0 \\ 0 & \inv y\end{pmatrix} \in \GL_{2n}^{(0)}(\widetilde{A/I})
    $$
    so by the corollary \ref{corol:lift by surjection} again, let $a \in \GL_n^{(0)}(\widetilde{A})$ and $b\in \GL_{2n}^{(0)}(\widetilde{A})$ be the lifts respectively. But then $u \coloneqq w \diag(a,b)$ is a lift of $\diag(y, \inv y)$. From the fact that $p_n$ commutes with $\diag(a,b)$, it is obtained that $up_n \inv u = wp_n \inv w$. Since we already show that the choice of lift doesn't matter, \ref{todo:(ii)} is checked.
    
    For $x,y \in \GL_n(\widetilde{A/I})$, suppose that $w$ is a lift of $\diag(x,\inv x)$ and $v$ is a lift of $\diag(y,\inv y)$. Notice that $\varpi \coloneqq \diag(w,v)$ is a lift of $\diag(x,y, \inv{x},\inv y)$, hence 
    \begin{equation*}
    \begin{array}{rl}
         \partial([x]_1[y]_1)= & \vspace{0.25cm}\left[ \varpi p_{2n}\inv\varpi\right]_0 - \left[p_{2n}\right]_0 
        \\= \vspace{0.25cm}& \left[ \begin{pmatrix}
            w & 0 \\ 0 & v
        \end{pmatrix} \begin{pmatrix}
            p_n & 0 \\ 0 & p_n
        \end{pmatrix}
        \inv{\begin{pmatrix}
            w & 0 \\ 0 & v
        \end{pmatrix}}
        \right]_0 - \left[ \begin{pmatrix}
            p_n & 0 \\ 0 & p_n
        \end{pmatrix}\right]_0
        \\= \vspace{0.25cm}&\left[\begin{pmatrix}
            wp_n\inv w & 0 \\ 0 & vp_n\inv v
        \end{pmatrix}\right]_0 - \left[\begin{pmatrix}
            p_n & 0 \\ 0 & p_n
        \end{pmatrix}\right]_0
        \\= & [wp_n\inv w]_0-[p_n]_0 + [vp_n\inv v]_0-[p_n]_0 = \partial [x]_1 + \partial[y]_1
    \end{array} 
    \end{equation*}
    Therefore, it is a group morphism as our final item \ref{todo:(iii)} assures. 
\end{proof}

\begin{definicao}[Index map in \ensuremath{K}-theory]
    Using construction \ref{contru: index map}, the group morphism so called \textit{index} map is given by
    \begin{equation*}
        \function{{\partial}{K_1(A/I)}{K_0(I)}{[x]_1}{[wp_n\inv w]_0 - [p_n]_0}}
    \end{equation*}
    whenever $x\in \GL_n(\widetilde{A/I})$ and $w \in \GL_{2n}^{(0)}(\widetilde{A})$ is a lift of $\diag(x,\inv x)$.
\end{definicao}

\begin{exemplo}
    In a unital $C^*$-algebra $A$, if a unitary\footnote{i.e. $u^*= \inv u$.} idempotent element $u$ in $\GL_n(A/I)$ lifts to $v\in \mathbb M_n(A)$, the element
    \begin{equation*}
        w \coloneqq \begin{pmatrix}
            v &  \Id_n - v^*v \\ \Id_n - vv^* & v^*
        \end{pmatrix}
    \end{equation*}
    is a lift for $\diag(u, \inv u)$. Therefore:
    \begin{eqnarray*}
        \partial [u]_1 &=& \left[wp_n \inv w\right]_0 - \left[p_n\right]_0 \\
        &=& \left[\left(\begin{smallmatrix}
            v &  \Id_n - v^*v \\ \Id_n - vv^* & v^*
        \end{smallmatrix}\right)
        \left(\begin{smallmatrix}
            \vphantom{\Id_n-v^*v}\Id_n & 0 \\ \vphantom{\Id_n-v^*v}0 & 0
        \end{smallmatrix}
        \right)\left(\begin{smallmatrix}
            v^* &  \Id_n - vv^* \\ \Id_n - v^*v & v
        \end{smallmatrix}\right)\right]_0 - \left[
            \left(\begin{smallmatrix}
                \vphantom{\Id_n-v^*v}\Id_n & 0 \\ \vphantom{\Id_n-v^*v}0 & 0
            \end{smallmatrix}\right)\right]_0 \\
            &=&  \left[\left(\begin{smallmatrix}
                vv* &  0 \\
                0 & \Id_n- vv^*
            \end{smallmatrix}\right)\right]_0 - \left[
                \left(\begin{smallmatrix}
                    \vphantom{\Id_n-v^*v}\Id_n & 0 \\ \vphantom{\Id_n-v^*v}0 & 0
                \end{smallmatrix}\right)\right]_0 \\
            &=& \left[\Id_n - v^*v\right]_0 - \left[\Id_n-vv^*\right]_0
    \end{eqnarray*}
    Notice that when $A = \mathscr B(H)$ and $I = \mathscr K(H)$, we dealing again with Fredholm operators living in the Calking algebra $\mathscr Q(H)$ and $\partial$ coincides with the Fredholm index. 
\end{exemplo}